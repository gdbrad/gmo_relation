\documentclass[prd,tightenlines,preprintnumbers,showpacs,superscriptaddress,notitlepage,nofootinbib,eqsecnum,floatfix,notitlepage]{revtex4-1}
% LOAD PREAMBLE
\documentclass[options]{style}
\usepackage{graphicx}  % needed for figures
\usepackage{dcolumn}   % needed for some tables
\usepackage{bm}        % for math
\usepackage{amssymb}   % for math
\usepackage{standalone}
\usepackage{enumitem}
\usepackage[pdftex]{color}
\usepackage{xcolor}
\usepackage{slashed}
\usepackage{booktabs}
\usepackage{multirow}
\usepackage{amsmath}
\usepackage{bbm}
\usepackage{stackrel}
\usepackage{rotating}
\usepackage{CJKutf8}
\usepackage{pifont}
\usepackage{mathtools}
\usepackage{hyperref}
\hypersetup{
    colorlinks=true,       % false: boxed links; true: colored links
    linkcolor=blue,          % color of internal links
    citecolor=blue,        % color of links to bibliography
    filecolor=blue,      % color of file links
    urlcolor=blue           % color of external links
}
\usepackage{simplewick}
% NM packages
\usepackage{float} % Forces placement of figures
\usepackage{tikz} % adds /foreach (looping) command
\usepackage{xspace}



% NEW COMMANDS
%\input{../command_list}

% NEW COMMANDS
\def\comment#1{\textbf{\color{red} COMMENT: #1}}
\def\addcite#1{{\color{red}CITE[#1]}}
% Refs
\def\eqref#1{{(\ref{#1})}}
\newcommand{\eqnref}[1]{Eq.~\eqref{#1}}
\newcommand{\Eqnref}[1]{Equation~\eqref{#1}}
\newcommand{\figref}[1]{Fig.~\ref{#1}}
\newcommand{\Figref}[1]{Figure~\ref{#1}}
\newcommand{\secref}[1]{Sec.~\ref{#1}}
\newcommand{\appref}[1]{App.~\ref{#1}}
% std for Table is to always use Table - strange
\newcommand{\tabref}[1]{Table~\ref{#1}}
\newcommand{\Tabref}[1]{Table~\ref{#1}}

% define colors
\definecolor{kngrey}{HTML}{A6AAA9}
\definecolor{knred}{HTML}{EC5D57}
\definecolor{knorange}{HTML}{F39019}
\definecolor{knyellow}{HTML}{F5D328}
\definecolor{kngreen}{HTML}{70BF41}
\definecolor{knblue}{HTML}{51A7F9}
\definecolor{knpurple}{HTML}{B36AE2}

\def\mc#1{{\mathcal #1}}
\def\ol#1{{\overline{#1}}}
\def\nxlo#1{{N$^{#1}$LO}}

\DeclareMathOperator{\st}{str}
\DeclareMathOperator{\tr}{tr}
\DeclareMathOperator{\Erfc}{Erfc}
\DeclareMathOperator{\Erf}{Erf}
\DeclareMathOperator{\Tr}{Tr}

\def\one{\ensuremath{\mathbbm{1}}}

% Greek Letters
\def\a{{\alpha}}
\def\b{{\beta}}
\def\d{{\delta}}
\def\D{{\Delta}}
\def\e{{\epsilon}}
\def\g{{\gamma}}
\def\G{{\Gamma}}
\def\k{{\kappa}}
\def\l{{\lambda}}
\def\L{{\Lambda}}
\def\m{{\mu}}
\def\n{{\nu}}
\def\w{{\omega}}
\def\O{{\Omega}}
\def\S{{\Sigma}}
\def\s{{\sigma}}
\def\t{{\tau}}
\def\th{{\theta}}
\def\x{{\xi}}

%slash's
\def\Dslash{D\hskip-0.65em /}
\def\dslash{{\partial\hskip-0.5em /}}
\def\vslash{{\rlap \slash v}}
\def\qbar{{\overline q}}

% Jargon
\def\CPT{{$\chi$PT}}
\def\QCPT{{Q$\chi$PT}}
\def\PQCPT{{PQ$\chi$PT}}
\def\tr{\text{tr}}
\def\str{\text{str}}
\def\diag{\text{diag}}
\def\vit{{\it v}}
\def\vD{\vit\cdot D}
\def\tb{{\tilde b}}

\def\gmo{{$\D_{\rm GMO}$}\xspace}


% code
\def\chroma{\texttt{Chroma}\xspace}
\def\quda{\texttt{QUDA}\xspace}

% \input{affiliations}

\newcount\hour \newcount\hourminute \newcount\minute
\hour=\time \divide \hour by 60
\hourminute=\hour \multiply \hourminute by 60
\minute=\time \advance \minute by -\hourminute
\newcommand{\mydate}{\ \today \ - \number\hour :\number\minute}


\begin{document}

\title{Notes on GMO relation}



\author{Andr\'{e}~Walker-Loud}

\date{\mydate}

\begin{abstract}
Notes for GMO project.
\end{abstract}
\maketitle
\tableofcontents
%-------------------------------------------------------------------------------------------------
%-------------------------------------------------------------------------------------------------

\section{Motivation}
The Gell-Mann--Okubo (GMO) baryon mass relation is
\begin{equation}\label{eq:D_GMO}
\D_{\rm GMO} = M_\L +\frac{1}{3}M_\S -\frac{2}{3} M_N -\frac{2}{3} M_\Xi\, .
\end{equation}
This quantity is interesting because it is a flavor-27 quantity, which means that the first non-vanishing correction to this relation in SU(3) baryon $\chi$PT arises from the one-loop sunset diagrams that are proportional to $m_q^{3/2}$ power, and thus non-analytic in the quark mass.
One then anticipates that this quantity will exhibit a more rapid change quark mass dependence than the baryon masses themselves.
While SU(3) baryon $\chi$PT is not a converging EFT in general, it may be that for this quantity, there is a convergence.

The GMO relation has been studied previously in the literature in Refs.~\cite{Beane:2006pt,Walker-Loud:2011yaf}.
Both of these results from from DWF on asqtad ensembles, at a single lattice spacing of $a\approx0.12$~fm and with $m_\pi \gtrsim300$~MeV.

We have results on 30 ensembles with four lattice spacings, $a\approx\{0.06, 0.09, 0.12, 0.15\}$~fm and seven pion masses in the range $130\lesssim m_\pi\lesssim400$~MeV.
Our aim is to explore how well SU(3) HB$\chi$PT describes the results, and, if the convergence pattern is healthy.




\section{Analysis Notes}

\subsection{Fitting the correlators to extract \gmo}
We examined fitting the individual baryon correlation functions and then forming the \gmo combination, but this yielded a result consistent with zero.
In contrast, if we construct the GMO correlation function
\begin{equation}\label{eq:C_GMO}
C_{\rm GMO}(t) = \frac{C_\L(t) C_\S^{1/3}(t)}{C_N^{2/3}(t) C_\Xi^{2/3}(t)}\, ,
\end{equation}
and examine its effective mass, this yields values that are several sigma different from zero.
We therefore need a more involved analysis of the correlation functions in order to extract precise values of \gmo on each ensemble.
Even if the two-point functions of each baryon are positive definite, we are not guaranteed that if we were to describe $C_{\rm GMO}(t)$ with a sum of exponentials, there overlap factors would all be positive.

However, we can probably fit this product correlation function in combination with the individual baryon correlation functions, in the following way, to obtain precise values of \gmo.
First, pull out the ground state overlap factors and energies from each single baryon correlation function, for which the correlator model is then
\begin{equation}
C_{\rm GMO}(t) =
    Z_{\rm GMO} e^{-\D_{\rm GMO} t}
    \frac{
        \left[1 + \sum_n r_{\L,n}e^{-\D_{\L,n}t}\right]
        \left[ 1 + \sum_n r_{\S,n}e^{-\D_{\S,n}t}\right]^{1/3}
        }{
        \left[1 + \sum_n r_{N,n}e^{-\D_{N,n}t}\right]^{2/3}
        \left[ 1 + \sum_n r_{\Xi,n}e^{-\D_{\Xi,n}t}\right]^{2/3}
        }\, ,
\end{equation}
where $\D_{\rm GMO}$ is the mass relation in \eqnref{eq:D_GMO}, $r_{B,n}$ are the ratios of the excited state to ground state overlap factor for baryon $B$ and $\D_{B,n}$ are the excited state mass splittings of baryon $B$ with the ground state mass.

In order to assess our ability to control the systematic uncertainty in the extraction of $\D_{\rm GMO}$, we use three different fit function-models.  In all cases, the GMO correlator is fit simultaneously with the four octet baryon correlators.
\begin{enumerate}
\item For the ground state GMO parameters, use the combination appearing in \eqnref{eq:C_GMO}:
\begin{align}
Z_{\rm GMO} &= \frac{A_{\L,0} A_{\S,0}^{1/3}}{A_{N,0}^{2/3} A_{\Xi,0}^{2/3}}\, ,
\nonumber\\
\D_{\rm GMO} &= M_\L +\frac{1}{3}M_\S -\frac{2}{3} M_N -\frac{2}{3} M_\Xi\, ;
\end{align}

\item Treat $\D_{\rm GMO}$ as an independent fit parameter, but take the overlap factors from the baryons;

\item Treat both $\D_{\rm GMO}$ and $Z_{\rm GMO}$ as independent fit parameters;
\end{enumerate}
If everything is working well, all three options will give similar results for $\D_{\rm GMO}$ and $Z_{\rm GMO}$.  It will be interesting to compare the fit quality and logGBF factors from the three fits.






\subsection{New and Old smearing on same ensemble}
On the a09m310 ensemble, I find that it is beneficial (reduced uncertainty) to fit all 4 correlators from the two different smearings.  To compare, I chose an optimal fit from fitting just the old data, as compared to the old \& new simultaneous fit.
The resulting analysis yields

\bigskip
\begin{tabular}{p{3.25in}p{3.25in}}
Old data only & New and Old smearing data\\
{\scriptsize
\begin{verbatim}
['proton_SS', 'proton_PS']
n_state = 3, t = [6, 20)
Least Square Fit:
  chi2/dof [dof] = 0.42 [22]    Q = 0.99    logGBF = 643.55

Parameters:
      proton_E_0       0.4931 (38)       [     0.495 (25) ]
     proton_zS_0   0.00002171 (56)       [ 0.0000220 (50) ]
     proton_zP_0     0.001138 (38)       [   0.00120 (25) ]
log(proton_dE_1)        -1.34 (52)       [     -1.27 (70) ]
     proton_zP_1       0.0014 (18)       [    0.0012 (36) ]
     proton_zS_1      5.7(8.7)e-06       [  0.000022 (44) ]
log(proton_dE_2)        -1.25 (63)       [     -1.27 (70) ]
     proton_zP_2      0.00004 (72)       [    0.0012 (36) ]
     proton_zS_2    0.0000193 (72)       [  0.000022 (44) ]
-----------------------------------------------------------
     proton_dE_1         0.26 (14)       [      0.28 (20) ]
     proton_dE_2         0.29 (18)       [      0.28 (20) ]

Settings:
  svdcut/n = 3.8e-06/1    tol = (1e-08,1e-10,1e-10*)
                   (itns/time = 11/0.0)
  fitter = scipy_least_squares    method = trf\end{verbatim}
}
&
{\scriptsize
\begin{verbatim}
['proton_S1S1', 'proton_PS1', 'proton_S2S2', 'proton_PS2']
n_state = 3, t = [9, 20)
Least Square Fit:
  chi2/dof [dof] = 0.74 [44]    Q = 0.9    logGBF = 1270.7

Parameters:
      proton_E_0       0.4927 (28)       [     0.495 (25) ]
    proton_zS2_0   0.00002163 (41)       [ 0.0000220 (50) ]
    proton_zS1_0    0.0003264 (72)       [  0.000340 (50) ]
     proton_zP_0     0.001137 (27)       [   0.00120 (25) ]
log(proton_dE_1)        -1.08 (20)       [     -1.27 (70) ]
     proton_zP_1      0.00210 (76)       [    0.0012 (36) ]
    proton_zS1_1     0.000343 (89)       [   0.00034 (68) ]
    proton_zS2_1      9.1(4.3)e-06       [  0.000022 (44) ]
log(proton_dE_2)        -1.55 (65)       [     -1.27 (70) ]
     proton_zP_2       0.0008 (15)       [    0.0012 (36) ]
    proton_zS1_2      0.00036 (19)       [   0.00034 (68) ]
    proton_zS2_2    -0.000019 (16)       [  0.000022 (44) ]
-----------------------------------------------------------
     proton_dE_1        0.340 (68)       [      0.28 (20) ]
     proton_dE_2         0.21 (14)       [      0.28 (20) ]

Settings:
  svdcut/n = 1.6e-06/1    tol = (1e-08,1e-10,1e-10*)
                   (itns/time = 27/0.1)
  fitter = scipy_least_squares    method = trf
\end{verbatim}
}
\end{tabular}


\begin{figure}
\begin{tabular}{cc}
\includegraphics[width=0.49\textwidth]{figures/a09m310_proton_E_0_stability_old}
&
\includegraphics[width=0.49\textwidth]{figures/a09m310_proton_E_0_stability_new_old}
\\
\includegraphics[width=0.49\textwidth]{figures/a09m310_m_proton_old}
&
\includegraphics[width=0.49\textwidth]{figures/a09m310_m_proton_new_old}
\\
(Old)& (New \& Old)
\end{tabular}
\caption{\label{fig:stability}
Stability analysis.
}
\end{figure}





\bibliography{c51_bib}



%-------------------------------------------------------------------------------------------------
%-------------------------------------------------------------------------------------------------

\end{document}
