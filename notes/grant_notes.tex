\documentclass[12pt,tightenlines, raggedbottom, prd, notitlepage]{revtex4-1}

\usepackage{amsmath}
\usepackage{amssymb}
\usepackage{braket}
\usepackage{booktabs}
\usepackage{hyperref} 
\usepackage{braket}

% Forces bold math in section/subsection/etc headings 
\makeatletter
\g@addto@macro\bfseries{\boldmath}
\makeatother
\def\mc#1{{\mathcal #1}}
\def\tr{\text{tr}}

\begin{document}


\title{GMO Relation}
\author{Andre Walker-Loud, Grant Bradley}
\maketitle
\section{Motivation}
The Gell-Mann--Okubo (GMO) baryon mass relation is
\begin{equation}
\Delta_{\rm GMO} = M_\lambda +\frac{1}{3}M_\sigma -\frac{2}{3} M_N -\frac{2}{3} M_\Xi\, .
\end{equation}
This quantity is interesting because it is a flavor-27 quantity, which means that the first non-vanishing correction to this relation in SU(3) baryon $\chi$PT arises from the one-loop sunset diagrams that are proportional to $m_q^{3/2}$ power, and thus non-analytic in the quark mass.
One then anticipates that this quantity will exhibit a more rapid change quark mass dependence than the baryon masses themselves.
While SU(3) baryon $\chi$PT is not a converging EFT in general, it may be that for this quantity, there is a convergence.

% The GMO relation has been studied previously in the literature in Refs.~\cite{Beane:2006pt,Walker-Loud:2011yaf}.
Both of these results from from DWF on asqtad ensembles, at a single lattice spacing of $a\approx0.12$~fm and with $m_\pi \gtrsim300$~MeV.

We have results on 30 ensembles with four lattice spacings, $a\approx\{0.06, 0.09, 0.12, 0.15\}$~fm and seven pion masses in the range $130\lesssim m_\pi\lesssim400$~MeV.
Our aim is to explore how well SU(3) HB$\chi$PT describes the results, and, if the convergence pattern is healthy.




\section{Analysis Notes}

\subsection{Fitting the correlators to extract GMO}
We examined fitting the individual baryon correlation functions and then forming the GMO combination, but this yielded a result consistent with zero.
In contrast, if we construct the GMO correlation function
\begin{equation}
C_{\rm GMO}(t) = \frac{C_\lambda(t) C_\sigma^{1/3}(t)}{C_N^{2/3}(t) C_\Xi^{2/3}(t)}\, ,
\end{equation}
and examine its effective mass, this yields values that are several sigma different from zero.
We therefore need a more involved analysis of the correlation functions in order to extract precise values of GMO on each ensemble.
Even if the two-point functions of each baryon are positive definite, we are not guaranteed that if we were to describe $C_{\rm GMO}(t)$ with a sum of exponentials, there overlap factors would all be positive.

However, we can probably fit this product correlation function in combination with the individual baryon correlation functions, in the following way, to obtain precise values of GMO.
First, pull out the ground state overlap factors and energies from each single baryon correlation function, for which the correlator model is then
\begin{equation}
C_{\rm GMO}(t) =
    \frac{A_{\lambda,0} A_{\sigma,0}^{1/3}}{A_{N,0}^{2/3} A_{\Xi,0}^{2/3}} e^{-\Delta_{\rm GMO} t}
    \frac{
        \left[1 + \sum_n r_{\lambda,n}e^{-\D_{\lambda,n}t}\right]
        \left[ 1 + \sum_n r_{\sigma,n}e^{-\D_{\sigma,n}t}\right]^{1/3}
        }{
        \left[1 + \sum_n r_{N,n}e^{-\D_{N,n}t}\right]^{2/3}
        \left[ 1 + \sum_n r_{\Xi,n}e^{-\D_{\Xi,n}t}\right]^{2/3}
        }\, ,
\end{equation}
where $A_{B,0}$ are the ground state overlap factors for baryon $B$,
$r_{B,n}$ are the ratio of the $n^{th}$ excited state overlap factor to the ground state and $\D_{B,n}$ is the $n^{th}$ excited state energy of baryon $B$.
If this product correlation function is fit simultaneously with each individual baryon two-point function, then the only new parameter to be constrained is \gmo.
We can go one step further and choose one of the ground state baryon masses to not be fit in addition to \gmo, but instead use the GMO formula, for example, one could replace $M_\Xi$ in the fit by
\begin{equation}
M_\Xi \rightarrow \frac{3}{2}M_\L + \frac{1}{2}M_\S +M_N -\D_{\rm GMO}\, .
\end{equation}

\end{document}